\documentclass[twocolumn]{article}
\usepackage{graphicx}
\usepackage{listings}
\usepackage{lmodern}
\usepackage{booktabs}
\usepackage{amsmath}
\usepackage{amssymb}
\usepackage{amsthm}
\usepackage{bbm}
\usepackage{multirow}
\usepackage{booktabs}
\usepackage{cancel}


%Math commands
\newcommand{\ev}[1]{\mathbb{E}\left(#1\right)}
\newcommand{\ov}[1]{\mathbb{O}(#1)}
\newcommand{\rmd}{\mathrm{d}}
\newcommand{\intg}[4]{\int_{#1}^{#2} \! #3 \, \rmd#4}
\newcommand{\der}[2]{\frac{\rmd^#1}{\rmd #2^#1}}


%Sectioning commands
\newcommand{\setsection}[1]{\setcounter{section}{#1}\addtocounter{section}{-1}\section{}}
\renewcommand{\thesubsection}{\alph{subsection})}

%Listings commands
\newcommand{\includecode}[1]{\lstinputlisting{#1}}
\newcommand{\code}[1]{\lstinline{#1}}

%Common settings
\lstset{language=R,basicstyle=\ttfamily}
\graphicspath{ {./img/} }
\author{Nakul Joshi}

\newcommand{\img}[1]{
\begin{figure}[!ht]
\centering
\includegraphics[width=0.4\textwidth]{#1}
\end{figure}
}


\title{MATH 308 Assignment 20:\\Power Function}
\date{April 22, 2014}

\begin{document}
\maketitle
\section{}
The null is rejected if $\bar X>\mu_0+\frac{z_{1-\alpha}}{\sqrt{n}}\equiv C$. But, $\bar X\N{\mu}{1^2/n}\implies \bar X=\mu+\frac{Z}{\sqrt{n}}$ \begin{align*}
\therefore\beta(\mu)
&=P(\bar X >C)\\
&=1-P(\bar X \le C)\\
&=1-P\left(\mu+\frac{Z}{\sqrt{n}}\le C\right)\\
&=1-P(Z\le(C-\mu)\sqrt{n})\\
&=1-\Phi((C-\mu)\sqrt{n})\\
&=1-\Phi(z_{1-\alpha}+(\mu_0-\mu)\sqrt{n})
\end{align*}
\section{}
Setting $\alpha=0.05,\mu_0=0$ and $n=100$, we get
\[
\beta(\mu)=1-\Phi(z_{0.95}-10\mu)
\]
\img{plot.pdf}
\section{}\begin{gather*}
0.8=1-\Phi(z_{0.95}+(0-1)\sqrt{n})\\
\implies z_{0.2}=z_{0.95}-\sqrt{n}\\
\implies n=(z_{0.95}-z_{0.2})^2\approx 7
\end{gather*}
\end{document}