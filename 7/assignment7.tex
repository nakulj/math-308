\documentclass[twocolumn]{article}
\usepackage{graphicx}
\usepackage{listings}
\usepackage{booktabs}
\usepackage{amsmath}
\usepackage{amssymb}
\usepackage{amsthm}
\usepackage{multirow}
\usepackage{booktabs}

\title{Math 308 Assignment 7\\Exercises 3.9}
\author{Nakul Joshi}

\newcommand{\setsection}[1]{\setcounter{section}{#1}\addtocounter{section}{-1}\section{}}
\renewcommand{\thesubsection}{\alph{subsection})}
\newcommand{\includecode}[1]{\lstinputlisting{#1}}
\newcommand{\code}[1]{\lstinline{#1}}
\newcommand{\intg}[4]{\int_{#1}^{#2} \! #3 \, \mathrm{d}#4}

\begin{document}
\lstset{language=R}
\graphicspath{ {./img/} }
\maketitle

\setsection{4}

The null hypothesis is that the difference in proportions is zero. However, performing the permutation test gave a $p$-value of 0.002, allowing us to reject the null at 1\% confidence. Thus, the difference in proportions is statistically significant.

\setsection{8}

The $p$-value is 1, which does not let us reject the null hypothesis that the presence of competition has no value on the height change of the seedlings.

\setsection{11}

\begin{description}
\item[Null Hypothesis] Voting preference is independent of age.
\item[Alternative hypothesis] Voting preference depends on age.

% Please remember to add \use{multirow} to your document preamble in order to suppor multirow cells
\begin{table}[!ht]
\centering
\begin{tabular}{lrrr}
\hline
\multirow{2}{*}{Age} & \multicolumn{3}{c}{Response}                                                     \\ \cline{2-4} 
                     & \multicolumn{1}{l}{For} & \multicolumn{1}{l}{Against} & \multicolumn{1}{l}{All}  \\ \hline
18-29                & 172                     & 52                          & \multicolumn{1}{|r}{224} \\
30-49                & 313                     & 103                         & \multicolumn{1}{|r}{416} \\
50+                  & 258                     & 119                         & \multicolumn{1}{|r}{377} \\ \cline{2-3}
All                  & 743                     & 274                         & 1017                     \\ \hline
\end{tabular}
\caption{Observed values}
\end{table}

\end{description}

Multiplying column marginal fractions by row marginal totals, we can get the expected values:

\begin{table}[!ht]
\centering
\begin{tabular}{@{}lrr@{}}
\toprule
\multirow{2}{*}{Age} & \multicolumn{2}{c}{Response}                          \\ \cmidrule(l){2-3} 
                     & \multicolumn{1}{l}{For} & \multicolumn{1}{l}{Against} \\ \midrule
18-29                & 164                     & 60                          \\
30-49                & 304                     & 112                         \\
50+                  & 275                     & 102                         \\ \bottomrule
\end{tabular}
\caption{Expected values}
\end{table}

Then, we calculate the $\chi^2$ test statitic:\\
$c=\sum_{i,j}^\text{all cells}\frac{\text{(observed}_{i,j}-\text{expected}_{i,j})^2}{\text{expected}_{i,j}}=6.33$

Under the null, $C$ follows a $\chi^2$ distribution with $(3-1)\times(2-1)=2$ degrees of freedom; i.e. $C\sim\chi^2_2$. So, the $p$-value is $P(C>c)=\intg{c}{\infty}{\frac{e^{-t/2}}{2}}{t}\approx0.042$.

Thus, we can reject the null at 5\% significance, but not at 1\% significance.

\setsection{13}

\setsection{17}

\setsection{19}

\setsection{22}

\setsection{25}

\setsection{29}

\end{document}