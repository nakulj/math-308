\documentclass[twocolumn]{article}
\usepackage{graphicx}
\usepackage{listings}
\usepackage{lmodern}
\usepackage{booktabs}
\usepackage{amsmath}
\usepackage{amssymb}
\usepackage{amsthm}
\usepackage{bbm}
\usepackage{multirow}
\usepackage{booktabs}
\usepackage{cancel}
\usepackage{hyperref}


%Math commands
\newcommand{\ev}[1]{\mathbb{E}\left(#1\right)}
\newcommand{\ov}[1]{\mathbb{O}(#1)}
\newcommand{\rmd}{\mathrm{d}}
\newcommand{\intg}[4]{\int_{#1}^{#2} \! #3 \, \rmd#4}
\newcommand{\der}[2]{\frac{\rmd^#1}{\rmd #2^#1}}


%Sectioning commands
\newcommand{\setsection}[1]{\setcounter{section}{#1}\addtocounter{section}{-1}\section{}}
\renewcommand{\thesubsection}{\alph{subsection})}
\renewcommand{\thesubsubsection}{
	\alph{subsection}.
	\arabic{subsubsection}
}

%Listings commands
\newcommand{\includecode}[1]{\lstinputlisting{#1}}
\newcommand{\code}[1]{\lstinline{#1}}

%Common settings
\lstset{language=R,basicstyle=\ttfamily}
\graphicspath{ {./img/} }
\author{Nakul Joshi}

\newcommand{\img}[1]{
\begin{figure}[!ht]
\centering
\includegraphics[width=0.4\textwidth]{#1}
\end{figure}
}

\newcommand{\N}[2]{\sim\mathcal{N}\left(#1,#2\right)}

\title{Math 308 Assignment 7\\Exercises 3.9}
\author{Nakul Joshi}


\newcommand{\nsub}[2]{N_{#1 #2}}

\newcommand{\ovi}{\ensuremath{\overline{\imath}}}
\newcommand{\ovj}{\ensuremath{\overline{\jmath}}}

\newcommand{\na}{\ensuremath{\nsub{i}{j}}}
\newcommand{\nb}{\ensuremath{\nsub{i}{\ovj}}}
\newcommand{\nc}{\ensuremath{\nsub{\ovi}{j}}}
\newcommand{\nd}{\ensuremath{\nsub{\ovi}{\ovj}}}

\newcommand{\sumr}{\ensuremath{\na+\nb}}
\newcommand{\sumc}{\ensuremath{\na+\nc}}
\newcommand{\sums}{\ensuremath{n}}

\begin{document}
\maketitle

\setsection{4}

The null hypothesis is that the difference in proportions is zero. However, performing the permutation test gave a $p$-value of 0.002, allowing us to reject the null at 1\% confidence. Thus, the difference in proportions is statistically significant.

\setsection{8}

The $p$-value is 1, which does not let us reject the null hypothesis that the presence of competition has no value on the height change of the seedlings.

\setsection{11}

\begin{description}
\item[Null Hypothesis] Voting preference is independent of age.
\item[Alternative hypothesis] Voting preference depends on age.

% Please remember to add \use{multirow} to your document preamble in order to suppor multirow cells
\begin{table}[!ht]
\centering
\begin{tabular}{lrrr}
\hline
\multirow{2}{*}{Age} & \multicolumn{3}{c}{Response}                                                     \\ \cline{2-4} 
                     & \multicolumn{1}{l}{For} & \multicolumn{1}{l}{Against} & \multicolumn{1}{l}{All}  \\ \hline
18-29                & 172                     & 52                          & \multicolumn{1}{|r}{224} \\
30-49                & 313                     & 103                         & \multicolumn{1}{|r}{416} \\
50+                  & 258                     & 119                         & \multicolumn{1}{|r}{377} \\ \cline{2-3}
All                  & 743                     & 274                         & 1017                     \\ \hline
\end{tabular}
\caption{Observed values}
\end{table}

\end{description}

Multiplying column marginal fractions by row marginal totals, we can get the expected values:

\begin{table}[!ht]
\centering
\begin{tabular}{@{}lrr@{}}
\toprule
\multirow{2}{*}{Age} & \multicolumn{2}{c}{Response}                          \\ \cmidrule(l){2-3} 
                     & \multicolumn{1}{l}{For} & \multicolumn{1}{l}{Against} \\ \midrule
18-29                & 164                     & 60                          \\
30-49                & 304                     & 112                         \\
50+                  & 275                     & 102                         \\ \bottomrule
\end{tabular}
\caption{Expected values}
\end{table}

Then, we calculate the $\chi^2$ test statistic:\\
$c=\sum_{i,j}^\text{all cells}\frac{\text{(observed}_{i,j}-\text{expected}_{i,j})^2}{\text{expected}_{i,j}}=6.33$

Under the null, $C$ follows a $\chi^2$ distribution with $(3-1)\times(2-1)=2$ degrees of freedom; i.e. $C\sim\chi^2_2$. So, the $p$-value is $P(C>c)=\intg{c}{\infty}{\frac{e^{-t/2}}{2}}{t}\approx0.042$.

Thus, we can reject the null at 5\% significance, but not at 1\% significance.

\setsection{13}

\subsection{} We are testing for homogenity since we want to know whether the distribution of fin ray counts differs from lake to lake.

\subsection{}
\begin{description}
\item[Null hypothesis] Fin ray distributions are the same from lake to lake.
\item[Alternative hypothesis] Fin ray distributions are different from lake to lake.
\end{description}

\begin{table}[h]
\begin{tabular}{lrrrrrrr}
\hline
\multirow{2}{*}{Habitat}                               & \multicolumn{7}{c}{Ray Count}                                                                                                                                                  \\ \cline{2-8} 
                                                       & \multicolumn{1}{l}{36} & \multicolumn{1}{l}{35} & \multicolumn{1}{l}{34} & \multicolumn{1}{l}{33} & \multicolumn{1}{l}{32} & \multicolumn{1}{l}{31} & \multicolumn{1}{l}{All}  \\ \hline
Guadalupe                                              & 14                     & 30                     & 42                     & 78                     & 33                     & 14                     & \multicolumn{1}{|r}{211} \\
Cedro                                                  & 11                     & 28                     & 53                     & 66                     & 27                     & 9                      & \multicolumn{1}{|r}{194} \\
\begin{tabular}[c]{@{}l@{}}San\\ Clemente\end{tabular} & 10                     & 17                     & 61                     & 53                     & 22                     & 10                     & \multicolumn{1}{|r}{173} \\ \cline{2-7}
All                                                    & 71                     & 110                    & 190                    & 230                    & 114                    & 64                     & 779                      \\ \hline
\end{tabular}
\end{table}

% Please remember to add \use{multirow} to your document preamble in order to suppor multirow cells
% Booktabs require to add \usepackage{booktabs} to your document preamble
\begin{table}[h]
\centering
\begin{tabular}{@{}lllllll@{}}
\toprule
\multirow{2}{*}{Habitat} & \multicolumn{6}{l}{Ray Count} \\ \cmidrule(l){2-7} 
                         & $\geq$36  & 35  & 34 & 33 & 32 & $\leq$31 \\ \midrule
Guadalupe                & 19  & 30  & 51 & 62 & 31 & 17 \\
Cedro                    & 18  & 27  & 47 & 57 & 28 & 16 \\
San Clemente             & 16  & 24  & 42 & 51 & 25 & 14 \\ \bottomrule
\end{tabular}
\caption{Expected Values}
\end{table}

$c=\sum_{i,j}^\text{all cells}\frac{\text{(observed}_{i,j}-\text{expected}_{i,j})^2}{\text{expected}_{i,j}}=41.77$, where $C\sim\chi^2_{10}$. So, $p=P(C>c)=\intg{c}{\infty}{
	\frac{
	t^{10/2-1}e^{-t/2}
	}{
	2^{10/2}\Gamma(10/2)
	}
}{t}
=\intg{c}{\infty}{
	\frac{
	t^{4}e^{-t/2}
	}{
	768
	}
}{t}=8\times10^{-6}$.
So, we can reject the null.

\setsection{17}
\subsection{}
\begin{table}[h]
\centering
\begin{tabular}{@{}lrr@{}}
\toprule
\multirow{2}{*}{Happiness} & \multicolumn{2}{c}{Gender} \\ \cmidrule(l){2-3} 
                           & Female        & Male       \\ \midrule
Not too happy              & 109           & 61         \\
Pretty happy               & 406           & 378        \\
Very happy                 & 205           & 210        \\ \bottomrule
\end{tabular}
\caption{Observed happiness against gender data}
\end{table}

\subsection{}

We get a $p$ value of 0.004, allowing us to reject the null hypothesis of independence at a 1\% significance level.

\setsection{19}
\subsection{} Let the elements of the contingency table be $o_{i,j}$, and let their corresponding row and column totals be $r_i$ and $c_j$ respectively. Further, let the total number of observations $\sum_{i,j} o_{i,j} = n$. The corresponding expected values are then $e_{i,j}=\frac{r_i c_j}{n}$, which gives the test statistic $c=\sum_{i,j}
	\frac{
		(o_{i,j}-e_{i,j})^2
	}{
		e_{i,j}
	}
$

However, if each element is multiplied by $k$, then $o_{i,j}$,$r_i$, $c_j$ and $n$ are each multiplied by $k$. So, the corresponding expected values become $e^*_{i,j}=\frac{k^2}{k}\times\frac{r_i c_j}{n}=k\times e_{i,j}$. The new test statistic $c^*=\sum_{i,j}
	\frac{
		(o^*_{i,j}-e^*_{i,j})^2
	}{
		e^*_{i,j}
	}
=\sum_{i,j}
\frac{k^2}{k}\times\frac{
		(o_{i,j}-e_{i,j})^2
	}{
		e_{i,j}
	}
=k\times c
$. Thus, the test statistic is also multiplied by $k$.

However, the marginal probabilities are unchanged since the $k$'s cancel on the row and overall totals. Further, the degrees of freedom are unaffected since they only depend upon $r$ and $c$.

\subsection{}

Originally, the $p$-value was $\intg{c}{\infty}{f(t;k)}{t}$, but the new $p$-value becomes $\intg{kc}{\infty}{f(t;k)}{t}$. Thus, the $p$ value reduces, since $k>1\implies kc>c$.

\setsection{22}

\subsection{}

% Booktabs require to add \usepackage{booktabs} to your document preamble
\begin{table}[!ht]
\centering
\begin{tabular}{@{}ll@{}}
\toprule
p   & q    \\ \midrule
0.2 & 16.1 \\
0.4 & 20.2 \\
0.6 & 23.8 \\
0.8 & 27.9 \\ \bottomrule
\end{tabular}
\end{table}

\newpage

\subsection{}
\begin{table}[!ht]
\centering
\begin{tabular}{@{}lrr@{}}
\toprule
\multicolumn{1}{c}{\multirow{2}{*}{Interval}} & \multicolumn{2}{c}{Counts}                          \\ \cmidrule(l){2-3} 
                                              & \multicolumn{1}{l}{Obs.} & \multicolumn{1}{l}{Exp.} \\ \midrule
\textless16.11                                & 16                       & 10                       \\
16.11--20.23                                  & 13                       & 10                       \\
20.23--23.77                                  & 9                        & 10                       \\
23.77--27.89                                  & 9                        & 10                       \\
\textgreater27.89                             & 3                        & 10                       \\ \bottomrule
\end{tabular}
\end{table}

\subsection{}

We get a $p$-value of 0.048, which does not let us reject the hypothesis that the data was drawn from a $N(22,7^2)$ distribution.

\setsection{25}

We get a $p$-value of 0.78, which does not let us reject the hypothesis that the numbers are uniformly distributed.

\setsection{29}

\subsection{}

Let \na be the element at row $i$ and column $j$. Then, since there are only two rows and two columns, let the remaining row and column indices be \ovi{} and \ovj{} respectively.
Then, the row total is $R_i=\sumr$ and the column total is $C_j=\sumc$. So, the expected value at $i,j$ is\begin{align*}
\breve{E}[\na]
&=\frac{R_i\times C_j}{\sums}	\\
&=\frac{(\sumr)(\sumc)}{\sums}
\end{align*}
This gives:\begin{align*}
&\na-\breve{E}[\na]\\
&=\na-\frac{(\sumr)(\sumc)}{\sums}\\
&=\frac{\na(\cancel{\na}+\cancel{\nb}+\cancel{\nc}+\nd)-(\cancel{\na^2}+\cancel{\na\nb}+\cancel{\na\nc}+\nb\nc)}{\sums}\\
&=\frac{\na\nd-\nb\nc}{\sums}\\
\end{align*}
Thus:\[
(\na-\breve{E}[\na])^2=\left(\frac{\na\nd-\nb\nc}{\sums}\right)^2
\]
This expression is symmetrical; i.e., it does not change by swapping $i$ and \ovi{} or $j$ and \ovj{}. So, it is equivalent to:\[
\left(\frac{N_{11}N_{22}-N_{21}N_{12}}{n}\right)^2
\]
which is independent of $i,j$ \qed.

\subsection{}

Call the previously obtained expression $k$. Then, \begin{align*}
C
&=kn/R_1C_1+kn/R_1C_2+kn/R_2C_1+kn/R_2C_2\\
&=kn\frac{R_1C_1+R_1C_2+R_2C_1+R_2C_2}{R_1R_2C_1C_2}\\
&=kn\frac{R_1(C_1+C_2)+R_2(C_1+C_2)}{R_1R_2C_1C_2}\\
&=kn\frac{n(R_1+R_2)}{R_1R_2C_1C_2}=k\frac{n^3}{R_1R_2C_1C_2}\\
&=\frac{(N_{11}N_{22}-N_{21}N_{12})^2}{n^2}\frac{n^3}{R_1R_2C_1C_2}\\
&=n(N_{11}N_{22}-N_{21}N_{12})^2/R_1R_2C_1C_2\qed
\end{align*}

\subsection{}
Via R, the expression is verified since both methods yield $C\approx0.0234$
\end{document}