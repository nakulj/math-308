\documentclass[twocolumn]{article}
\usepackage{graphicx}
\usepackage{listings}
\usepackage{booktabs}
\usepackage{amsmath}
\usepackage{amssymb}
\usepackage{amsthm}

\title{Math 308 Assignment 7\\Exercises 3.9}
\author{Nakul Joshi}

\newcommand{\setsection}[1]{\setcounter{section}{#1}\addtocounter{section}{-1}\section{}}
\renewcommand{\thesubsection}{\alph{subsection})}
\newcommand{\includecode}[1]{\lstinputlisting{#1}}
\newcommand{\code}[1]{\lstinline{#1}}
\newcommand{\intg}[4]{\int_{#1}^{#2} \! #3 \, \mathrm{d}#4}

\begin{document}
\lstset{language=R}
\graphicspath{ {./img/} }
\maketitle

\setsection{4}

The null hypothesis is that the difference in proportions is zero. However, performing the permutation test gave a p-value of 0.002, allowing us to reject the null at 1\% confidence. Thus, the difference in proportions is statistically significant.

\setsection{8}

\setsection{11}

\setsection{13}

\setsection{17}

\setsection{19}

\setsection{22}

\setsection{25}

\setsection{29}

\end{document}