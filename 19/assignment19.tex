\documentclass[twocolumn]{article}
\usepackage{graphicx}
\usepackage{listings}
\usepackage{lmodern}
\usepackage{booktabs}
\usepackage{amsmath}
\usepackage{amssymb}
\usepackage{amsthm}
\usepackage{bbm}
\usepackage{multirow}
\usepackage{booktabs}
\usepackage{cancel}


%Math commands
\newcommand{\ev}[1]{\mathbb{E}\left(#1\right)}
\newcommand{\ov}[1]{\mathbb{O}(#1)}
\newcommand{\rmd}{\mathrm{d}}
\newcommand{\intg}[4]{\int_{#1}^{#2} \! #3 \, \rmd#4}
\newcommand{\der}[2]{\frac{\rmd^#1}{\rmd #2^#1}}


%Sectioning commands
\newcommand{\setsection}[1]{\setcounter{section}{#1}\addtocounter{section}{-1}\section{}}
\renewcommand{\thesubsection}{\alph{subsection})}

%Listings commands
\newcommand{\includecode}[1]{\lstinputlisting{#1}}
\newcommand{\code}[1]{\lstinline{#1}}

%Common settings
\lstset{language=R,basicstyle=\ttfamily}
\graphicspath{ {./img/} }
\author{Nakul Joshi}

\newcommand{\img}[1]{
\begin{figure}[!ht]
\centering
\includegraphics[width=0.4\textwidth]{#1}
\end{figure}
}

\usepackage{mhchem}

\title{MATH 308 Assignment 19:\\Exercises 9.7}
\date{April 22, 2014}

\begin{document}
\maketitle

\setsection{7}
\subsection{}
$\rho\approx0.50$
\subsection{}
\begin{table}[h]\centering
\begin{tabular}{@{}lrrrr@{}}
\toprule
  & A    & B     & C    & D     \\ \midrule
$\bar X$ & 4.88 & 8.03  & 2.06 & 10.64 \\
$\bar Y$ & 9.91 & 13.24 & 6.89 & 14.68 \\ \bottomrule
\end{tabular}
\end{table}
\subsection{}
\img{7.pdf}
$\rho=0.99$, which is much higher than the result from (a).

\setsection{9}
\begin{gather*}
g(a,b)=\Sigma(y_i-\hat y_i)^2\\
\implies \frac{\partial g}{\partial a}
=\Sigma \left(2(y_i-\hat y_i)\frac{\partial (y_i-(a+bx_i))}{\partial a}\right)\\
\implies 0=-2\Sigma (y_i-\hat y_i)\\
\implies \Sigma (y_i-\hat y_i)=0 \qed
\end{gather*}


\setsection{10}
\subsection{}

\begin{gather*}
r=\frac{ss_{xy}}{\sqrt{ss_x ss_y}}, b=\frac{ss_{xy}}{ss_x}\\
\begin{aligned}
\Sigma(\hat y_i-\bar y)^2&=\Sigma(a+bx_i-(a+b\bar x))^2\\
&=b^2\Sigma(x_i-\bar x)^2=b^2 ss_x
=\frac{ss_{xy}^2}{ss_x}\\
&=r^2ss_y\\
\implies s_{\hat y}
&=\sqrt{
	\frac{\Sigma(\hat y_i-\bar y)^2}
	{n-1}}
=r\sqrt{\frac{ss_y}{n-1}}\\
&=r\sqrt{
	\frac{\Sigma(y_i-\bar y)^2}
	{n-1}
}=rs_y\qed
\end{aligned}
\end{gather*}

\subsection{}
\begin{gather*}
\bar e=\Sigma(y_i-\hat y_i)/n=0\\
\implies s_e=\sqrt{\frac{\Sigma(y_i-\hat y_i-0)^2}{n-1}}\\
\begin{aligned}
\implies &(n-1)s_e^2\\
&=\Sigma(y_i-\hat y_i)^2\\
&=\Sigma(y_i-\bar y+\bar y-\hat y_i)^2\\
&=\Sigma(y_i-\bar y)^2+\Sigma(\bar y-\hat y_i)^2\\&-2\Sigma(y_i-\bar y)(\hat y_i-\bar y)\\
&=ss_y+r^2ss_y-2\Sigma(y_i\hat y_i-\cancel{y_i\bar y}-\hat y_i\bar y+\cancel{{\bar y}^2})\\
&=ss_y(1+r^2)-2\Sigma(\hat y_i\cancel{(y_i-\bar y)})\\
\end{aligned}\\
\implies s_e=\sqrt{\frac{(1+r^2)ss_y}{n-1}}=\sqrt{1+r^2}s_y\qed
\end{gather*}

\setsection{11}
\subsection{}
\begin{gather*}
b=r\frac{s_w}{s_h}\approx 1.6 a=\bar w-b\bar h\approx 20.1\\
\implies \hat w=a+bh\approx 20.1+1.6h
\end{gather*}
\subsection{}
$\hat w(5 \text{ ft})=\hat w(60 \text{ in})=116.5\text{ pounds}$
\subsection{}
$r^2\approx0.56$, which means that the model explains 56\% of the variability of the weights of the individuals.

\setsection{17}
\subsection{}
Correlation is $\approx0.35$
\img{17a.pdf}
\subsection{}
$y=-2380+149x$
\subsection{}
The slope is interpreted as `a week increase in gestation period is correlated with a 149g increase in weight'.

The $r^2\approx0.12$ is interpreted as `the linear model explains 12\% of the variation in weights'.
\subsection{}
\img{17d.pdf}
The residual plot shows a curve peaking at week 39. The linear model is therefore inappropriate.



\setsection{18}
\subsection{} 457.4 g
\subsection{} $1-\alpha=0.95\implies 1-\alpha/2=0.975\\\implies t\equiv t_{n-2,1-\alpha/2}=t_{1007,0.975}\approx 1.96$\\
$\therefore\text{ 95\% CI: }\hat \beta \pm \frac{tS}{\sqrt{ss_x}}=149\pm 24.8\\=(124.2,173.8)$

\setsection{21}
\subsection{}
\img{21a.pdf}
The data is strongly linear, with a correlation of over 0.99.
\subsection{}
\[y=-3279.593+1.826x\]
\newpage
\subsection{}
\img{21c.pdf}
At first, it appears that there is a pattern to the residuals, decreasing until year 2000 and then increasing again. However, looking at the scale of the $y$-axis we see that the largest residual is only ~2, but the \ce{CO2} Levels are of order ~350.

Thus, the apparent pattern can be put down to random error and ignored, making a linear model appropriate.

\end{document}