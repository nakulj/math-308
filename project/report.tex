\documentclass[twocolumn]{article}
\usepackage{graphicx}
\usepackage{listings}
\usepackage{lmodern}
\usepackage{booktabs}
\usepackage{amsmath}
\usepackage{amssymb}
\usepackage{amsthm}
\usepackage{bbm}
\usepackage{multirow}
\usepackage{booktabs}
\usepackage{cancel}


%Math commands
\newcommand{\ev}[1]{\mathbb{E}\left(#1\right)}
\newcommand{\ov}[1]{\mathbb{O}(#1)}
\newcommand{\rmd}{\mathrm{d}}
\newcommand{\intg}[4]{\int_{#1}^{#2} \! #3 \, \rmd#4}
\newcommand{\der}[2]{\frac{\rmd^#1}{\rmd #2^#1}}


%Sectioning commands
\newcommand{\setsection}[1]{\setcounter{section}{#1}\addtocounter{section}{-1}\section{}}
\renewcommand{\thesubsection}{\alph{subsection})}

%Listings commands
\newcommand{\includecode}[1]{\lstinputlisting{#1}}
\newcommand{\code}[1]{\lstinline{#1}}

%Common settings
\lstset{language=R,basicstyle=\ttfamily}
\graphicspath{ {./img/} }
\author{Nakul Joshi}

\newcommand{\img}[1]{
\begin{figure}[!ht]
\centering
\includegraphics[width=0.4\textwidth]{#1}
\end{figure}
}


\title{MATH 308 Project Report\\VARC Attendance Data Analysis}
\date{April 30, 2014}

\begin{document}
\maketitle
\thispagestyle{empty}

\abstract{The Viterbi Academic Resource Center (VARC) holds thirty minute, one-on-one sessions to assist students with introductory engineering coursework. VARC has recently been trying to solve an appointment utilisation problem: on some days, there are not enough sessions available, while on others, tutors sit at their offices with nothing to do. For my project, I have tried to analyse historical attendance data, to see if the situation can be improved.}

\section{Hypotheses}
\begin{description}
	\item[$H_0$] The session utilisation rate does not vary significantly between days of the week. In other words, the availability of sessions has already been optimally distributed throughout the week.
	\item[$H_A$] The session utilisation rate varies between days of the week.
\end{description}

\section{Data Testing}
VARC data from the year 2013 was analysed, listing over 4,000 sessions along with details such as day of week, and whether or not an appointment was cancelled.

For the purposes of the study, a session was considered `used' simply if it had been booked, even if the student subsequently did not show. A session that had been cancelled was not counted as `used', unless it had been subsequently re-booked.

A few appointments were conducted on weekends; these were dropped. The data was then grouped by day of the week and a proportions test was carried out across the groups. The essential \code{R} code used for the same has been listed below:

\begin{lstlisting}
usage=0
avail=0
for(i in datasets) {
	data<-read.csv(i)
	data$USED=
		(data$CONDUCTED | data$NOSHOW)
	usage<-usage+
		tapply(data$USED,data$DOW,sum)
	avail<-avail+
		tapply(data$USED,data$DOW,length)
}
prop.test(usage,avail)
\end{lstlisting}
\vspace{-1.5em}
\section{Results}
\vspace{-1.25em}
\begin{lstlisting}
5-sample test for equality of proportions
without continuity correction

data:  usage out of avail
X-squared = 37.4144, df = 4, p-value = 1.48e-07
alternative hypothesis: two.sided
sample estimates:
prop 1    prop 2    prop 3    prop 4    prop 5 
0.6558287 0.7615063 0.7131783 0.6369231 0.6597582 
\end{lstlisting}

\section{Conclusion}
The data clearly allows us to reject the null. It seems that VARC could benefit by moving some sessions to Tuesday and Wednesday; however, cross-validation will be required for the same.

\end{document}