\documentclass[twocolumn]{article}
\usepackage{url}
\usepackage{graphicx}
\usepackage{listings}
\usepackage{lmodern}
\usepackage{booktabs}
\usepackage{amsmath}
\usepackage{amssymb}
\usepackage{amsthm}
\usepackage{bbm}
\usepackage{multirow}
\usepackage{booktabs}
\usepackage{cancel}


%Math commands
\newcommand{\ev}[1]{\mathbb{E}\left(#1\right)}
\newcommand{\ov}[1]{\mathbb{O}(#1)}
\newcommand{\rmd}{\mathrm{d}}
\newcommand{\intg}[4]{\int_{#1}^{#2} \! #3 \, \rmd#4}
\newcommand{\der}[2]{\frac{\rmd^#1}{\rmd #2^#1}}


%Sectioning commands
\newcommand{\setsection}[1]{\setcounter{section}{#1}\addtocounter{section}{-1}\section{}}
\renewcommand{\thesubsection}{\alph{subsection})}

%Listings commands
\newcommand{\includecode}[1]{\lstinputlisting{#1}}
\newcommand{\code}[1]{\lstinline{#1}}

%Common settings
\lstset{language=R,basicstyle=\ttfamily}
\graphicspath{ {./img/} }
\author{Nakul Joshi}

\newcommand{\img}[1]{
\begin{figure}[!ht]
\centering
\includegraphics[width=0.4\textwidth]{#1}
\end{figure}
}


\title{MATH 308 Assignment S:\\Confounding Effects in Studies}
\date{March 26, 2014}

\begin{document}
\maketitle
\begin{description}
\item[Article being reviewed] `Coffee linked to premature death, study says'
\footnote{August 16, 2013, Fox News. \url{http://www.foxnews.com/health/2013/08/16/coffee-linked-to-premature-death-study-says/}.} 
\item[Study cited] `Association of Coffee Consumption With All-Cause and Cardiovascular Disease Mortality'\footnote{
	Liu et. al., Mayo Clinic Proceedings - October 2013 (Vol. 88, Issue 10, Pages 1066-1074, DOI: 10.1016/j.mayocp.2013.06.020). \url{http://www.mayoclinicproceedings.org/article/S0025-6196(13)00578-8/abstract}
	}
\end{description}

The article referenced suggests that reducing coffee consumption can lead to greater life expectancy; in other words, greater coffee consumption is causally related to early deaths.

However, the study was observational rather than a randomised controlled trial- the data was gathered by interviewing several individuals, gathering data on coffee consumption habits, and then tracking deaths across a period of 17 years. This is only sufficient to establish an \emph{association} between coffee consumption and early deaths, due to the possibility of confounding effects. For example, it is known that there is a relationship between cofee-drinking and smoking\footnote{Klesges, Robert C., JoAnne W. Ray, and Lisa M. Klesges. `Caffeinated coffee and tea intake and its relationship to cigarette smoking: an analysis of the Second National Health and Nutrition Examination Survey (NHANES II).' Journal of substance abuse 6.4 (1994): 407-418.}: if smokers are more likely to die due to lung disease, this would make the study appear to show the same of coffee drinkers. Thus, cutting coffee consumption might not necessarily help increase lifespan, if there is a deeper underlying cause for the increased mortality rate.

Further, the study only tracked `heavy' coffee consumption of over four cups a day. Therefore, this data might not at all be relevant to the typical coffee consumer, who might consume far less. This reveals another possible confounded effect: people who moderate their coffee consumption for reasons of health may also be making other lifestyle choices that actually do improve life expectancy.

Despite this, the study authors claim that ``it seems appropriate to suggest that younger people avoid heavy coffee consumption". The general tone of the article also seems to espouse moderating coffee consumption, without any mention of the various deficits of the study.
\end{document}