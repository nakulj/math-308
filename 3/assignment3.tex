\documentclass[twocolumn]{article}
\usepackage{amsmath}
\usepackage{amssymb}
\usepackage{amsthm}

\title{MATH 308 Exercises 1.11}
\author{Nakul Joshi}
\renewcommand{\thesection}{}
\renewcommand{\thesubsection}{\alph{subsection})}

\begin{document}
\maketitle

\section*{2}
\subsection{} Observational study.
\subsection{} No, because of the possibility of confounded effects.

\section*{3}
\subsection{} Observational study.
\subsection{} No.
\subsection{} No, because the study sample was a selected non-random sample of the population.

\section*{5}
Number of unique subsets of size $N$ is $\binom{N}{n}$.\\
The number of unique subsets that include a given individual is $\binom{N-1}{n-1}$.\\
$\therefore$ Required probability= $\binom{N}{n}/\binom{N-1}{n-1}=\frac{n}{N} \qed$

This formula does not change with the individual. Therefore, by symmetry, every person has an equal chance of being in the group $\qed$.

\section*{6}
\subsection{} From (5), with $N=10^8$ and $n=10^3$, required probability $p= n/N=10^{-5}$.
\subsection{} Probability of not being in any of 2000 independently chosen samples = $(1-p)^{2000} \approx 98\%$.
\subsection{} A half-chance of being in at least one sample implies a half-chance of being in no samples. So, if $t$ samples are chosen, \begin{align*}
&				& q^t		&= 0.5\\
&\implies & t \log q&=\log 0.5 \\
&\implies & t			&=\frac{\log 0.5}{\log(1-10^{-5})}\\
&&&=69315
\end{align*}
\end{document}