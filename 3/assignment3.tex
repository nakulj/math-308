\documentclass[twocolumn]{article}
\usepackage{amsmath}
\usepackage{amssymb}
\usepackage{cancel}
\usepackage{amsthm}

\title{MATH 308 Assignment 3\\Exercises 1.11}
\author{Nakul Joshi}
\newcommand{\setsection}[1]{\setcounter{section}{#1}\addtocounter{section}{-1}\section{}}
\renewcommand{\thesubsection}{\alph{subsection})}

\begin{document}
\maketitle

\setsection{2}
\subsection{} Observational study.
\subsection{} No, because it is possible that people without dementia are predisposed to drinking alcohol.

\setsection{3}
\subsection{} Observational study.
\subsection{} No, because it is likely that students who already use marijuana would be interested in listening to music that had references to marijuana.
\subsection{} No, because the study sample was not collected in a controlled, randomized fashion. Further, it excludes students who are not in high school.

\setsection{5}
Number of unique subsets of size $N$ is $\binom{N}{n}$.\\
The number of unique subsets that include a given individual is $\binom{N-1}{n-1}$.\\
$\therefore$ Required probability\begin{align*}
	&=\binom{N-1}{n-1}/\binom{N}{n}\\
	&=\frac{(N-1)!}{\cancel{(N-n)!}(n-1)!}\times\frac{\cancel{(N-n)!}n!}{N!}\\
	&=\frac{(N-1)!}{N!}\times\frac{n!}{(n-1)!}\\
	&=\frac{n}{N} \qed
\end{align*}

This formula does not change with the individual. Therefore, by symmetry, every person has an equal chance of being in the group. $\qed$

\setsection{6}
\subsection{} From (5), with $N=10^8$ and $n=10^3$, required probability $p= n/N=10^{-5}$.
\subsection{} Probability of not being in any of 2000 independently chosen samples = $(1-p)^{2000} \approx 98\%$.
\subsection{} A half-chance of being in at least one sample implies a half-chance of being in no samples. So, if $t$ samples are chosen, \begin{align*}
&				& q^t		&= 0.5\\
&\implies & t \log q&=\log 0.5 \\
&\implies & t			&=\frac{\log 0.5}{\log(1-10^{-5})}\\
&&&=69315
\end{align*}
\end{document}