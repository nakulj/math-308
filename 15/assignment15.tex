\documentclass[twocolumn]{article}
\usepackage{graphicx}
\usepackage{listings}
\usepackage{lmodern}
\usepackage{booktabs}
\usepackage{amsmath}
\usepackage{amssymb}
\usepackage{amsthm}
\usepackage{bbm}
\usepackage{multirow}
\usepackage{booktabs}
\usepackage{cancel}


%Math commands
\newcommand{\ev}[1]{\mathbb{E}\left(#1\right)}
\newcommand{\ov}[1]{\mathbb{O}(#1)}
\newcommand{\rmd}{\mathrm{d}}
\newcommand{\intg}[4]{\int_{#1}^{#2} \! #3 \, \rmd#4}
\newcommand{\der}[2]{\frac{\rmd^#1}{\rmd #2^#1}}


%Sectioning commands
\newcommand{\setsection}[1]{\setcounter{section}{#1}\addtocounter{section}{-1}\section{}}
\renewcommand{\thesubsection}{\alph{subsection})}

%Listings commands
\newcommand{\includecode}[1]{\lstinputlisting{#1}}
\newcommand{\code}[1]{\lstinline{#1}}

%Common settings
\lstset{language=R,basicstyle=\ttfamily}
\graphicspath{ {./img/} }
\author{Nakul Joshi}

\newcommand{\img}[1]{
\begin{figure}[!ht]
\centering
\includegraphics[width=0.4\textwidth]{#1}
\end{figure}
}


\newcommand{\str}[1]{\overbrace{a_#1a_#1\ldots a_#1}^{k_#1 \text{times}}}



\title{MATH 308 Assignment 15:\\Exercises 5.10}
\date{March 26, 2014}

\begin{document}
\maketitle

\setsection{5}

\subsection{} Suppose that cookie orders are represented such that, for example, an order of 1 sugar, 2 chocolate chips, 0 oatmeal, 0 peanut butter, and 2 ginger snaps is given by $|c|cc|cc|$. Then, two of the bars are fixed, giving the string $|???????|$, where the nine $?$ symbols are to be replaced by five $c$'s and two $|$'s. If we choose $5$ distinct positions in which to place the $c$'s, the positions of the remaining $|$'s is fixed. Thus, the number of possible cookie orders is $\binom{9}{5}$.

\subsection{}
More generally, we can use `cookie types' to represent occurances of particular elements in the chosen sets. Then, fixing the position of two $|$'s leaves $n+1-2=n-1$ $|$'s and  $n$ $c$'s, for a total of $2n-1$ symbols. Picking element members requires $n$ position choices, giving the total number of set choices as $\binom{2n-1}{n}$.

\subsection{}
For a sample $S$ of size $n$, each bootstrap sample represents the choice of $n$ elements from $S$ with replacement. Thus, the number of bootstrap samples is as above $\binom{2n-1}{n}$.

\setsection{6}
\subsection{}
The number of bootstrap samples with $k_1$ $a_1$'s, $k_2$ $a_2$'s \ldots $k_n$ $a_n$'s is the same as the number of permutations of the string \[
\underbrace{\str{1}\str{2}\ldots \str{n}}_{n \text{ symbols}}
\]
Using the formula for string permuations with repeated characters, this yields \[\frac{n!}{k_1!k_2!\ldots k_n!}\]
which, by definition, equals
\[\binom{n}{k_1,k_2\ldots k_n}\qed\]

\subsection{}
A bootstrap sample with $k_i$ $a_i$'s requires choosing $k_i$ of $n$ positions, after which the remaining $n-k_i$ positions can be filled by any of the remaining $n-1$ elements. Thus, the number of such samples is: \[(n-1)^{n-k_i}\binom{n}{k_i}\]
The total number of bootstrap samples is simply $n^n$. Thus, the required probability is \[
\frac{(n-1)^{n-k_i}}{n^n}\binom{n}{k_i}
\]

\setsection{8}
\subsection{}

\img{8sampleMeans200.pdf}
\begin{table}[!ht]	
\centering
\begin{tabular}{@{}rrrrrr@{}}
\toprule
Min.  & 1st Qu. & Median & Mean  & 3rd Qu. & Max.  \\ \midrule
18.21 & 19.58   & 20.02  & 20.02 & 20.44   & 22.49 \\ \bottomrule
\end{tabular}
\end{table}

\img{8sampleMeans50.pdf}
\begin{table}[!ht]	
\centering
\begin{tabular}{@{}rrrrrr@{}}
\toprule
Min.  & 1st Qu. & Median & Mean  & 3rd Qu. & Max.  \\ \midrule
16.19 & 19.26   & 20.03  & 20.03 & 20.87   & 24.43 \\ \bottomrule
\end{tabular}
\end{table}

\img{8sampleMeans10.pdf}
\begin{table}[!ht]	
\centering
\begin{tabular}{@{}rrrrrr@{}}
\toprule
Min.  & 1st Qu. & Median & Mean  & 3rd Qu. & Max.  \\ \midrule
12.56 & 17.87   & 19.99  & 20.05 & 21.89   & 33.20 \\ \bottomrule
\end{tabular}
\end{table}

\newpage

\subsection{}
\begin{table}[!ht]
\centering	
\begin{tabular}{@{}rrr@{}}
\toprule
\multicolumn{1}{l}{n} & \multicolumn{1}{l}{Mean} & \multicolumn{1}{l}{SD} \\ \midrule
200                   & 20.00                    & 8.94                   \\
50                    & 19.98                    & 8.98                   \\
10                    & 19.98                    & 9.00                   \\ \bottomrule
\end{tabular}
\end{table}

\subsection{}
\img{8bootstrap200.pdf}
\img{8bootstrap50.pdf}
\img{8bootstrap10.pdf}

\begin{table}[!ht]
\centering	
\begin{tabular}{@{}rrr@{}}
\toprule
\multicolumn{1}{l}{n} & \multicolumn{1}{l}{Mean} & \multicolumn{1}{l}{SE} \\ \midrule
200                   & 20.00                    & 0.65                   \\
50                    & 19.98                    & 1.25                   \\
10                    & 19.98                    & 2.82                   \\ \bottomrule
\end{tabular}
\end{table}

\stepcounter{subsection}
\subsection{}
Increasing sample size reduces the bootstrap standard error.

\clearpage

\setsection{9}
\newcommand{\ninefig}[1]{
\begin{figure}[!ht]
\centering
  \begin{minipage}[b]{0.24\textwidth}
    \includegraphics[width=\textwidth]{9even#1.pdf}
  \end{minipage} \begin{minipage}[b]{0.24\textwidth}
    \includegraphics[width=\textwidth]{9odd#1.pdf}
  \end{minipage}
\end{figure}
}

\ninefig{14}
\ninefig{36}
\ninefig{200}
\ninefig{10000}

The evenness of $N$ does not seem to affect the histograms, but increasing $N$ reduces the spread.

\newpage

\setsection{10}
\subsection{}
\img{10a.pdf}
\begin{table}[h]
\begin{tabular}{@{}rrrrrr@{}}
\toprule
Min. & 1st Qu. & Median & Mean  & 3rd Qu. & Max.    \\ \midrule
1.00 & 5.00    & 14.20  & 78.08 & 55.50   & 1550.00 \\ \bottomrule
\end{tabular}
\end{table}
The data appears to be highly concentrated between 0 and 500, with several outliers that skew the mean.

\subsection{}
\img{10b.pdf}
Mean: 78.03\\
95\% confidence interval: 55.0 -- 104.61
\subsection{}
Bias is 0.37, which is $\approx$ 3\% of the SE.
\setsection{11}
\img{11.pdf}
Trimmed Mean: 17.90\\
95\% confidence interval: 13.67 -- 23.30\\
Bias is 0.27, which is $\approx$ 11\% of the SE.
The new test statistic has a smaller 95\% confidence interval at the cost of higher bias.
\setsection{12}
\setsection{14}

\end{document}
