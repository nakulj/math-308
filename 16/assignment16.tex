\documentclass[twocolumn]{article}
\usepackage{graphicx}
\usepackage{listings}
\usepackage{lmodern}
\usepackage{booktabs}
\usepackage{amsmath}
\usepackage{amssymb}
\usepackage{amsthm}
\usepackage{bbm}
\usepackage{multirow}
\usepackage{booktabs}
\usepackage{cancel}
\usepackage{hyperref}


%Math commands
\newcommand{\ev}[1]{\mathbb{E}\left(#1\right)}
\newcommand{\ov}[1]{\mathbb{O}(#1)}
\newcommand{\rmd}{\mathrm{d}}
\newcommand{\intg}[4]{\int_{#1}^{#2} \! #3 \, \rmd#4}
\newcommand{\der}[2]{\frac{\rmd^#1}{\rmd #2^#1}}


%Sectioning commands
\newcommand{\setsection}[1]{\setcounter{section}{#1}\addtocounter{section}{-1}\section{}}
\renewcommand{\thesubsection}{\alph{subsection})}
\renewcommand{\thesubsubsection}{
	\alph{subsection}.
	\arabic{subsubsection}
}

%Listings commands
\newcommand{\includecode}[1]{\lstinputlisting{#1}}
\newcommand{\code}[1]{\lstinline{#1}}

%Common settings
\lstset{language=R,basicstyle=\ttfamily}
\graphicspath{ {./img/} }
\author{Nakul Joshi}

\newcommand{\img}[1]{
\begin{figure}[!ht]
\centering
\includegraphics[width=0.4\textwidth]{#1}
\end{figure}
}

\newcommand{\N}[2]{\sim\mathcal{N}\left(#1,#2\right)}

\newcommand{\phat}{\ensuremath{\hat{p}}}
\newcommand{\lhat}{\hat{\lambda}}

\title{MATH 308 Assignment 16:\\Exercises 6.4}
\date{April 10, 2014}

\begin{document}
\maketitle

\setsection{1}
We know that $X\sim\mathcal{B}(n,p)\implies f_X(x)=\binom{n}{x}p^x(1-p)^{n-x}$. Thus, the probability of obtaining the value $X$ can be written as a function of the parameter $p$:
\begin{align*}
f(p)=\binom{n}{X}p^X(1-p)^{n-X}
\end{align*}
\begin{align*}
\implies f'/\binom{n}{X}&=Xp^{X-1}(1-p)^{n-X}\\
&-p^X(n-X)(1-p)^{n-X-1}
\end{align*}
We can then obtain the MLE \phat{} by setting $f'=0$:\begin{align*}
&Xp^{X-1}(1-p)^{n-X}=p^X(n-X)(1-p)^{n-X-1}\\
&\implies \phat(n-X)=X(1-\phat)\\
&\implies \phat n-\cancel{\phat X}=X-\cancel{X\phat}\\
&\implies \phat=X/n \qed
\end{align*}

\setsection{2}
$X\sim\mathcal{P}(\lambda)\implies f_X=\frac{\lambda^xe^{-\lambda}}{x!}$. So, the likelihood of obtaining the sample $\{x_1,x_2,\ldots x_n\}$ is $L(\lambda)=\prod_{i=1}^nf_X(x_i)$, since the elements of the sample are i.i.d.\begin{align*}
L&=\prod_{i=1}^n\frac{\lambda^{x_i}e^{-\lambda}}{x_i!}\\
&=\frac{\lambda^{\Sigma x_i}e^{-n\lambda}}{\prod (x_i!)}
\end{align*}
\begin{align*}
\implies L'/\prod (x_i!)&= \left(\lambda^{\Sigma x_i}e^{-n\lambda}\right)'\\
&= \Sigma x_i \lambda^{\Sigma x_i-1} e^{-n\lambda}
+ \lambda^{\Sigma x_i}e^{-n\lambda}(-n)\\
&=e^{-n\lambda}\lambda^{\Sigma x_i-1}\left( \Sigma x_i - n\lambda \right)
\end{align*}
Setting $L'=0$, we get $\lhat$:\begin{align*}
&0=\Sigma x_i - n\lhat\\
&\implies\lhat= \Sigma x_i/n = \overline{x}\qed
\end{align*}
\setsection{4}
\setsection{5}
\setsection{10}
\setsection{12}
\setsection{14}
\setsection{16}
\setsection{25}
\setsection{27}
\setsection{34}
\setsection{36}

\end{document}