\documentclass[twocolumn]{article}

\usepackage{graphicx}
\usepackage{listings}
\usepackage{lmodern}
\usepackage{booktabs}
\usepackage{amsmath}
\usepackage{amssymb}
\usepackage{amsthm}
\usepackage{bbm}
\usepackage{multirow}
\usepackage{booktabs}
\usepackage{cancel}
\usepackage{hyperref}


%Math commands
\newcommand{\ev}[1]{\mathbb{E}\left(#1\right)}
\newcommand{\ov}[1]{\mathbb{O}(#1)}
\newcommand{\rmd}{\mathrm{d}}
\newcommand{\intg}[4]{\int_{#1}^{#2} \! #3 \, \rmd#4}
\newcommand{\der}[2]{\frac{\rmd^#1}{\rmd #2^#1}}


%Sectioning commands
\newcommand{\setsection}[1]{\setcounter{section}{#1}\addtocounter{section}{-1}\section{}}
\renewcommand{\thesubsection}{\alph{subsection})}
\renewcommand{\thesubsubsection}{
	\alph{subsection}.
	\arabic{subsubsection}
}

%Listings commands
\newcommand{\includecode}[1]{\lstinputlisting{#1}}
\newcommand{\code}[1]{\lstinline{#1}}

%Common settings
\lstset{language=R,basicstyle=\ttfamily}
\graphicspath{ {./img/} }
\author{Nakul Joshi}

\newcommand{\img}[1]{
\begin{figure}[!ht]
\centering
\includegraphics[width=0.4\textwidth]{#1}
\end{figure}
}

\newcommand{\N}[2]{\sim\mathcal{N}\left(#1,#2\right)}

\newcommand{\sfrac}[2]{
	\frac{#1}{
	2^{#2} #2!
	}
}

\title{Math 308 Assignment 9\\Moments of the Standard Normal}
\author{Nakul Joshi}
\date{March 6, 2014}

\begin{document}
\maketitle

\section{Moment Generating Function}
From the definition of the m.g.f.,\begin{align*}
M_Z(t)&=\ev{e^{tZ}}\\
&=\intg{-\infty}{\infty}{
	e^{tx} \frac{1}{\sqrt{2\pi}} e^{-x^2/2}
	}{x}\\
&=\intg{-\infty}{\infty}{
	\frac{1}{\sqrt{2\pi}} e^{-(x-t)^2/2} e^{t^2/2}
	}{x}\\
&=e^{t^2/2}\intg{-\infty}{\infty}{
	f_X(x)
	}{x}\\
\text{where $X \sim \mathcal{N}(t,0)$}\\
&=e^{t^2/2}
\end{align*}

\section{Moments}
Using the series expansion of the exponential\[
e^x=\sum_{i=0}^\infty \frac{x^i}{i!}
\]

We observe that \begin{align*}
M_Z(t)	&= \sum_{i=0}^\infty \frac{(t^2/2)^i}{i!} \\
		&= 1+\frac{t^2}{2}+\sfrac{t^4}{2}+\sfrac{t^6}{3}+\sfrac{t^8}{4} \ldots\\
M_Z'(t)	&= \frac{2t}{2} + \sfrac{4t^3}{2} + \sfrac{6t^5}{3}+\sfrac{8t^7}{4} \ldots\\
M_Z''(t)&= 1+\sfrac{4\times 3t^2}{2}+\sfrac{6\times 5t^4}{3}+\sfrac{8\times 7}{4} \ldots
\end{align*}

\newpage

This tells us that, for odd values of $n$, $M_Z^{(n)}(t)$ will be an expression of the form $t\times (\text{a polynomial})$. Thus, for these values, $M_Z^{(n)}(0)$ is 0.

For the even values of $n$, we can see that $M_Z^{(n)}(t)$ is simply the $n$th derivative of the $\frac{n}{2}$th term in the series expansion of $M_Z(t)$, plus an expression of the form $t\times (\text{a polynomial})$. The latter expression always evaluates to 0 at $t=0$, so we can drop it. To get the $n$th derivative of the $i$th term: \begin{align*}
	T_i &=\sfrac{t^{2i}}{i}\\
	\der{n}{t} T_i &= \sfrac{(2i)(2i-1)(2i-2)\ldots(2i-n+1)}{i} t^{2i-n}\\
	&=\frac{(2i)!}{(2i-n)!}\sfrac{1}{i}t^{2i-n}\\
	\text{Setting $i$ to $n/2$,}\\
	\ev{Z^n}&=\frac{n!}{0!}\sfrac{1}{(n/2)}t^0\\
	&=\frac{n!}{(n/2)!}\frac{1}{2^{n/2}}\\
\end{align*}

We can then compute and tabulate the moments:
\begin{table}[h]
\centering
\begin{tabular}{@{}llllllll@{}}
\toprule
$n$   & 0 & 1 & 2 & 3 & 4 & 5 & 6  \\ \midrule
$\ev{Z^n}$ & 1 & 0 & 1 & 0 & 3 & 0 & 15 \\ \bottomrule
\end{tabular}
\end{table}

\end{document}