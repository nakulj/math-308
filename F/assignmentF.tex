\documentclass[twocolumn]{article}
\usepackage{graphicx}
\usepackage{listings}
\usepackage{lmodern}
\usepackage{booktabs}
\usepackage{amsmath}
\usepackage{amssymb}
\usepackage{amsthm}
\usepackage{bbm}
\usepackage{multirow}
\usepackage{booktabs}
\usepackage{cancel}
\usepackage{hyperref}


%Math commands
\newcommand{\ev}[1]{\mathbb{E}\left(#1\right)}
\newcommand{\ov}[1]{\mathbb{O}(#1)}
\newcommand{\rmd}{\mathrm{d}}
\newcommand{\intg}[4]{\int_{#1}^{#2} \! #3 \, \rmd#4}
\newcommand{\der}[2]{\frac{\rmd^#1}{\rmd #2^#1}}


%Sectioning commands
\newcommand{\setsection}[1]{\setcounter{section}{#1}\addtocounter{section}{-1}\section{}}
\renewcommand{\thesubsection}{\alph{subsection})}
\renewcommand{\thesubsubsection}{
	\alph{subsection}.
	\arabic{subsubsection}
}

%Listings commands
\newcommand{\includecode}[1]{\lstinputlisting{#1}}
\newcommand{\code}[1]{\lstinline{#1}}

%Common settings
\lstset{language=R,basicstyle=\ttfamily}
\graphicspath{ {./img/} }
\author{Nakul Joshi}

\newcommand{\img}[1]{
\begin{figure}[!ht]
\centering
\includegraphics[width=0.4\textwidth]{#1}
\end{figure}
}

\newcommand{\N}[2]{\sim\mathcal{N}\left(#1,#2\right)}

\title{MATH 308 Assignment F:\\Sample Final Problem}
\date{May 2, 2014}

\begin{document}
\maketitle

\newcommand{\sol}{\textbf{Solution.}}

\emph{Benford's Law}, also known as the \emph{First-Digit Law}, describes the distribution of the first digit of numbers in many real-life sources of data. In base 4, the law states that the leading digit $d \in \{1,2,3\}$ is described by the probability mass function:\[
P(d)=\log_{4}\left(1+\frac{1}{d}\right)
\]

A financial auditor is trying to apply Benford's Law to verify the accounts of a client. She knows that if the numbers on the account report deviate significantly from the Benford's Law distribution, the numbers are likely to be fudged. So, she decides to gather a random sample of $n$ numbers from the report, and test them.

\subsection{}
Describe a test statistic she can use, remembering to state the null and alternative hypotheses about the statistic.\\
\sol She can use a chi-squared test, by tabulating the frequencies of occurences of each of ${1,2,3}$ as the first digit. Using the pdf, she can also obtain expected values for the same frequencies. The chi-squared statistic $t$ thus obtained is the required test statistc. The null hypothesis is that the data follows Benford's law, i.e. $t\sim\chi^2_2$. The alternative is that it does not.
\newpage
\subsection{}
Assuming that she is testing at the significance level $\alpha$, what is the critical region?\\
\sol The critical region $c$ is given by \[c=[0,2-a)\cap (2+a,\inf) \], where $P(2-a\le t\le 2+a)=1-\alpha$. $a$ is obtained from a table of chi-square values (2 is the mean of the chi-square on 2 degrees of freedom).
\subsection{}
Write a program, using either \code{R} or pseudocode, to perform the test on a dataset.\\
\sol \begin{enumerate}
\item Read list of numbers in base 3
\item Extract observed values by reading first digits
\item Extract expected values using pmf.
\item Perform chi-squared test
\item Compare $p$-value from chi-squared test against $\alpha$.
\end{enumerate}
\end{document}