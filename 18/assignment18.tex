\documentclass[twocolumn]{article}
\usepackage{graphicx}
\usepackage{listings}
\usepackage{lmodern}
\usepackage{booktabs}
\usepackage{amsmath}
\usepackage{amssymb}
\usepackage{amsthm}
\usepackage{bbm}
\usepackage{multirow}
\usepackage{booktabs}
\usepackage{cancel}


%Math commands
\newcommand{\ev}[1]{\mathbb{E}\left(#1\right)}
\newcommand{\ov}[1]{\mathbb{O}(#1)}
\newcommand{\rmd}{\mathrm{d}}
\newcommand{\intg}[4]{\int_{#1}^{#2} \! #3 \, \rmd#4}
\newcommand{\der}[2]{\frac{\rmd^#1}{\rmd #2^#1}}


%Sectioning commands
\newcommand{\setsection}[1]{\setcounter{section}{#1}\addtocounter{section}{-1}\section{}}
\renewcommand{\thesubsection}{\alph{subsection})}

%Listings commands
\newcommand{\includecode}[1]{\lstinputlisting{#1}}
\newcommand{\code}[1]{\lstinline{#1}}

%Common settings
\lstset{language=R,basicstyle=\ttfamily}
\graphicspath{ {./img/} }
\author{Nakul Joshi}

\newcommand{\img}[1]{
\begin{figure}[!ht]
\centering
\includegraphics[width=0.4\textwidth]{#1}
\end{figure}
}


\title{MATH 308 Assignment 18:\\Exercises 8.5}
\date{April 22, 2014}

\begin{document}
\maketitle

\setsection{4}
p=$P(X>=20)=\text{\code{1-ppois(19,lambda=15)}}\approx 12.5\%$. This implies that more than 10\% of the months will have a birth rate as extreme as the one observed, which means that we cannot reject the null ($\lambda=15$) even at the 5\% significance level.

\setsection{6}
\subsection{} Running the test gives a $p$-value of $10^{-4}$. This lets us reject the null hypothesis (that the difference in means is zero) at the $1\%$ significance level.
\subsection{} The random assignment of seedlings to plots tends to remove confounding effects. Thus, the result hints at a causal relationship between presence of competition and diameter change.

\setsection{11}
Running the chi-squared test gives a $p$-value of $0.012$, allowing us to reject the null at the 5\% significance level.

\setsection{14}
A Type I error would be incorrectly concluding that the drug is effective when in fact it is not. This would lead to to patients wasting money on a useless drug.

A Type II error would be failing to conclude that the drug is effective, when in fact it is. This would lead to a waste of the research money put into the drug, since the pharmaceutical company would then not be able to sell it.

\setsection{16}
Under the null, $\bar X\N{25}{16/30}$. Thus $Z=\frac{\bar X-25}{4/\sqrt{30}}$, where $Z$ is a standard normal random variable. Thus, the null is rejected if $Z\ge z_{1-\alpha}
\implies \frac{\bar X-25}{4/\sqrt{30}}\ge z_{0.95}\implies \bar X\ge 25+\frac{4z_{0.95}}{\sqrt{30}}\equiv C$.

But $\bar X \N{27}{16/\sqrt{30}}\implies Z=\frac{\bar X-27}{4/\sqrt{30}}$. So, \begin{align*}
	1-\beta&=P(\bar X\ge C|\mu=27)\\
	&=P\left(\frac{4Z}{\sqrt{30}}+27\ge C\right)\\
	&=P\left(Z\ge\frac{\sqrt{30}(C-27)}{4}\right)\equiv P(Z \ge C_1)\\
	&=1-P(Z<C_1)\approx 86.3\%
\end{align*}

\setsection{17}
The null is rejected if  $\bar X \ge \mu_0+\frac{z_{1-\alpha}\sigma}{\sqrt{n}}\equiv C$, where $\mu_0=1,\alpha=0.01\text{ and } \sigma=0.3$. So,
\begin{gather*}
\begin{aligned}
1-\beta&=P(\bar X\ge C|\mu=\mu_1)\\
&=P\left(\frac{Z\sigma}{\sqrt{n}}+\mu_1\ge C\right)\\
&=P\left(Z\ge\frac{\sqrt{n}(C-\mu_1)}{\sigma}\right)\equiv P(Z\ge C_1)\\
&=1-P(Z<C_1)\\
\end{aligned}\\
\implies \beta =P(Z<C_1)\\
\implies z_\beta=C_1=\frac{\sqrt{n}(C-\mu_1)}{\sigma}\\
\implies \frac{z_\beta \sigma}{\sqrt{n}}+\mu_1=C=\frac{z_{1-\alpha}\sigma}{\sqrt{n}}+\mu_0\\
\implies n=\left(
	\frac{\sigma(z_\beta-z_{1-\alpha})}{\mu_0-\mu_1}
\right)^2=8
\end{gather*}


\setsection{18}
\begin{align*}
n&=\left(
	\frac{\sigma(z_\beta-z_{1-\alpha})}{\mu_0-\mu_1}
\right)^2=1
\end{align*}


\setsection{25}
$f(x)=\lambda e^{-\lambda x}\implies F(x)=\intg{0}{x}{\lambda e^{-\lambda t}}{t}=1-e^{-\lambda t}$
\subsection{}
\begin{gather*}
\begin{aligned}
F_{X,\min}(x)&=\prod_{i=1}^{15} P(X_i\le x)\\
&=(1-F(x))^{15}\\
&=e^{-15\lambda x}
\end{aligned}\\
\begin{aligned}
\implies \alpha &=P(X_{\min}\ge 1|\lambda=1/5)\\
&=1-\left.F_{X,\min}(1)\right|_{\lambda=\frac{1}{5}}\\
&=1-e^{-3}\approx 95\%
\end{aligned}
\end{gather*}

\subsection{}
\begin{align*}
1-\beta&=P(X_{\min} \ge 1|\lambda=1/25)\\
&=1-\left.F_{X,\min}(1)\right|_{\lambda=\frac{1}{25}}\\
&=1-e^{-3/5}\approx 45\%
\end{align*}

\setsection{36}
The most powerful test simply always rejects the null. Thus, the critical region is $X_1,X_2\in [0,n]$.


\setsection{37}
Under the null, the likelihood is maximised at $\mu=\mu_0$ and $\sigma^2=\Sigma(X_i-\mu_0)^2/n\equiv\hat\sigma^2$.
Thus, the likelihood under the null is\[
L_{H,0}=\left(\frac{1}{\hat\sigma\sqrt{2\pi}}
\right)^n\exp\left(
\frac{-\Sigma(X_i-\mu_0)^2}{2\hat\sigma^2}
\right)=\frac{e^{-n/2}}{(\hat\sigma\sqrt{2\pi})^n}
\]
To maximise the likelihood, we simply use $\mu=\bar X$ and $\sigma=\sqrt{\Sigma(X_i-\bar X)^2/n}$, giving \[
L=\left(\frac{1}{S\sqrt{2\pi}}
\right)^n\exp\left(
\frac{-\Sigma(X_i-\bar X)^2}{2S^2}
\right)=\frac{e^{-n/2}}{(S\sqrt{2\pi})^n}
\]
The likelihood ratio is therefore $(S/\hat\sigma)^n$. So, we reject if\begin{gather*}
\left(\frac{\Sigma (X_i-\bar X)^2}{\Sigma(X_i-\mu_0)^2}\right)^{n/2}\le c \implies \frac{\Sigma (X_i-\bar X)^2}{\Sigma(X_i-\mu_0)^2} \le c_1\\
\begin{aligned}
\implies c_1&\le \frac{\Sigma(X_i-\mu_0)^2}{\Sigma (X_i-\bar X)^2}
=\frac{\Sigma(X_i-\bar X+\bar X-\mu_0)^2}{\Sigma (X_i-\bar X)^2}\\
&\le1+\frac{n(\bar X-\mu_0)^2-2(\bar X-\mu_0)\cancel{\Sigma(X_i-\bar X)}}{\Sigma (X_i-\bar X)^2}\\
&\le 1+\frac{n}{\Sigma (X_i-\bar X)^2} (\bar X-\mu_0)^2\\
\end{aligned}\\
\implies c_2\le\sqrt{\frac{n}{\Sigma (X_i-\bar X)^2}}(\bar X-\mu_0)
=\frac{\bar X-\mu_0}{S/\sqrt{n}}
\end{gather*}
which is a $t$-statistic. Thus, the test is a one-sided $t$-test. $\qed$
\end{document}