\documentclass[twocolumn]{article}
\usepackage{graphicx}
\usepackage{listings}
\usepackage{lmodern}
\usepackage{booktabs}
\usepackage{amsmath}
\usepackage{amssymb}
\usepackage{amsthm}
\usepackage{bbm}
\usepackage{multirow}
\usepackage{booktabs}
\usepackage{cancel}
\usepackage{hyperref}


%Math commands
\newcommand{\ev}[1]{\mathbb{E}\left(#1\right)}
\newcommand{\ov}[1]{\mathbb{O}(#1)}
\newcommand{\rmd}{\mathrm{d}}
\newcommand{\intg}[4]{\int_{#1}^{#2} \! #3 \, \rmd#4}
\newcommand{\der}[2]{\frac{\rmd^#1}{\rmd #2^#1}}


%Sectioning commands
\newcommand{\setsection}[1]{\setcounter{section}{#1}\addtocounter{section}{-1}\section{}}
\renewcommand{\thesubsection}{\alph{subsection})}
\renewcommand{\thesubsubsection}{
	\alph{subsection}.
	\arabic{subsubsection}
}

%Listings commands
\newcommand{\includecode}[1]{\lstinputlisting{#1}}
\newcommand{\code}[1]{\lstinline{#1}}

%Common settings
\lstset{language=R,basicstyle=\ttfamily}
\graphicspath{ {./img/} }
\author{Nakul Joshi}

\newcommand{\img}[1]{
\begin{figure}[!ht]
\centering
\includegraphics[width=0.4\textwidth]{#1}
\end{figure}
}

\newcommand{\N}[2]{\sim\mathcal{N}\left(#1,#2\right)}

\title{MATH 308 Assignment 18:\\Exercises 8.5}
\date{April 22, 2014}

\begin{document}
\maketitle

\setsection{4}
p=$P(X>=20)=\text{\code{1-ppois(19,lambda=15)}}\approx 12.5\%$. This implies that more than 10\% of the months will have a birth rate as extreme as the one observed, which means that we cannot reject the null ($\lambda=15$) even at the 5\% significance level.

\setsection{6}
\subsection{} Running the test gives a $p$-value of $10^{-4}$. This lets us reject the null hypothesis (that the difference in means is zero) at the $1\%$ significance level.
\subsection{} The random assignment of seedlings to plots tends to remove confounding effects. Thus, the result hints at a causal relationship between presence of competition and diameter change.

\setsection{11}
Running the chi-squared test gives a $p$-value of $0.012$, allowing us to reject the null at the 5\% significance level.

\setsection{14}
A Type I error would be incorrectly concluding that the drug is effective when in fact it is not. This would lead to to patients wasting money on a useless drug.

A Type II error would be failing to conclude that the drug is effective, when in fact it is. This would lead to a waste of the research money put into the drug, since the pharmaceutical company would then not be able to sell it.

\setsection{16}
Under the null, $\bar X\N{25}{16/30}$. Thus $Z=\frac{\bar X-25}{4/\sqrt{30}}$, where $Z$ is a standard normal random variable. Thus, the null is rejected if $Z\ge z_{1-\alpha}
\implies \frac{\bar X-25}{4/\sqrt{30}}\ge z_{0.95}\implies \bar X\ge 25+\frac{4z_{0.95}}{\sqrt{30}}\equiv C$.

But $\bar X \N{27}{16/\sqrt{30}}\implies Z=\frac{\bar X-27}{4/\sqrt{30}}$. So, \begin{align*}
	1-\beta&=P(\bar X\ge C|\mu=27)\\
	&=P\left(\frac{4Z}{\sqrt{30}}+27\ge C\right)\\
	&=P\left(Z\ge\frac{\sqrt{30}(C-27)}{4}\right)\equiv P(Z \ge C_1)\\
	&=1-P(Z<C_1)\approx 86.3\%
\end{align*}

\setsection{17}
\setsection{18}
\setsection{25}
\setsection{36}
\setsection{37}

\end{document}