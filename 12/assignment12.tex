\documentclass[twocolumn]{article}
\usepackage{graphicx}
\usepackage{listings}
\usepackage{lmodern}
\usepackage{booktabs}
\usepackage{amsmath}
\usepackage{amssymb}
\usepackage{amsthm}
\usepackage{bbm}
\usepackage{multirow}
\usepackage{booktabs}
\usepackage{cancel}
\usepackage{hyperref}


%Math commands
\newcommand{\ev}[1]{\mathbb{E}\left(#1\right)}
\newcommand{\ov}[1]{\mathbb{O}(#1)}
\newcommand{\rmd}{\mathrm{d}}
\newcommand{\intg}[4]{\int_{#1}^{#2} \! #3 \, \rmd#4}
\newcommand{\der}[2]{\frac{\rmd^#1}{\rmd #2^#1}}


%Sectioning commands
\newcommand{\setsection}[1]{\setcounter{section}{#1}\addtocounter{section}{-1}\section{}}
\renewcommand{\thesubsection}{\alph{subsection})}
\renewcommand{\thesubsubsection}{
	\alph{subsection}.
	\arabic{subsubsection}
}

%Listings commands
\newcommand{\includecode}[1]{\lstinputlisting{#1}}
\newcommand{\code}[1]{\lstinline{#1}}

%Common settings
\lstset{language=R,basicstyle=\ttfamily}
\graphicspath{ {./img/} }
\author{Nakul Joshi}

\newcommand{\img}[1]{
\begin{figure}[!ht]
\centering
\includegraphics[width=0.4\textwidth]{#1}
\end{figure}
}

\newcommand{\N}[2]{\sim\mathcal{N}\left(#1,#2\right)}

\title{MATH 308 Assignment 12:\\Sequences in Coin Flips}
\date{March 26, 2014}

\begin{document}
\maketitle

\section{Expected Number of Occurences}
For each coin flip $i$, let $X_i$ be the event that the sequence $01111111$ occurs starting at position $i$. Then:\[
\ev{\mathbbm{1}(X_i)}=\begin{cases}
	2^{-8},& \text{if } 1 \le i \le 93 \\
	0,& \text{if } i>93
\end{cases}
\]
So, the expected number of occurences is:\begin{align*}
N	&= \ev{\sum_{i=1}^{100} \mathbbm{1}(X_i)}\\
	&= \sum_{i=1}^{100} \ev{\mathbbm{1}(X_i)}\\
	&= \sum_{i=1}^{93} \ev{\mathbbm{1}(X_i)}\\
	&= 93/2^8 \approx 0.363
\end{align*}

\newpage

\section{Probability of Occurence}
Using the result above, we can approximate the coin-flipping experiment as a Poisson process with rate parameter $\lambda=93/2^8$.

The requisite probability can then be calculated as \code{1-ppois(0,93/2^8)}, which comes out to $\approx{0.30}$, which is slightly under the result from Assignment 2.
\end{document}