\documentclass[twocolumn]{article}
\usepackage{graphicx}
\usepackage{listings}
\usepackage{lmodern}
\usepackage{booktabs}
\usepackage{amsmath}
\usepackage{amssymb}
\usepackage{amsthm}
\usepackage{bbm}
\usepackage{multirow}
\usepackage{booktabs}
\usepackage{cancel}
\usepackage{hyperref}


%Math commands
\newcommand{\ev}[1]{\mathbb{E}\left(#1\right)}
\newcommand{\ov}[1]{\mathbb{O}(#1)}
\newcommand{\rmd}{\mathrm{d}}
\newcommand{\intg}[4]{\int_{#1}^{#2} \! #3 \, \rmd#4}
\newcommand{\der}[2]{\frac{\rmd^#1}{\rmd #2^#1}}


%Sectioning commands
\newcommand{\setsection}[1]{\setcounter{section}{#1}\addtocounter{section}{-1}\section{}}
\renewcommand{\thesubsection}{\alph{subsection})}
\renewcommand{\thesubsubsection}{
	\alph{subsection}.
	\arabic{subsubsection}
}

%Listings commands
\newcommand{\includecode}[1]{\lstinputlisting{#1}}
\newcommand{\code}[1]{\lstinline{#1}}

%Common settings
\lstset{language=R,basicstyle=\ttfamily}
\graphicspath{ {./img/} }
\author{Nakul Joshi}

\newcommand{\img}[1]{
\begin{figure}[!ht]
\centering
\includegraphics[width=0.4\textwidth]{#1}
\end{figure}
}

\newcommand{\N}[2]{\sim\mathcal{N}\left(#1,#2\right)}


\newcommand{\sm}[1]{\mu_{\overline{#1}}}
\newcommand{\sv}[1]{\sigma^2_{\overline{#1}}}
\newcommand{\fmin}{f_{\text{min}}}
\newcommand{\Fmin}{F_{\text{min}}}
\newcommand{\fmax}{f_{\text{max}}}
\newcommand{\Fmax}{F_\text{max}}
\newcommand{\lo}{\lambda_1}
\newcommand{\lt}{\lambda_2}

\title{MATH 308 Assignment 8:\\Exercises 4.4}

\begin{document}
\maketitle

\setsection{1}
\img{1.pdf}
\noindent Population median: 5.5\\
Mean of sample medians: 5.7

\newpage

\setsection{2}
\img{2.pdf}
\noindent Mean of minimums: 4.8\\


\setsection{6}
From the CLT, the sample mean is approximately distributed normally with $\sm{X}=\mu=48$ and $\sv{X}=\frac{\sigma^2}{n}=9^2/20=2.7$. Thus the probability of the sample mean being greater than 51 is \code{1-pnorm(51,mean=48,sd=sqrt(2.7))} which evaluates to $\approx 3\%.$

\setsection{9}
The mean $\mu$ of the distribution is $\intg{2}{6}{f(x)}{x}=4$. The variance $\sigma^2$ is $\intg{2}{6}{(x-\mu)^2f(x)}{x}=2.4$. By CLT, the distribution of the sample mean $\overline{X}$ is approximated by $\mathcal{N}(\mu,\frac{\sigma^2}{n})=\mathcal{N}(4,3/305)$. Thus, $P(\overline{X}>=4.2)$ is \code{1-pnorm(4.2,mean=4,sd=sqrt(2.4))}, which evaluates to $\approx 2\%$.

\setsection{10}
By CLT, the number of people with degrees is approximately normally dustributed with $\mu=np$ and $\sigma^2=np(1-p)$. Thus, the probability of between 220 and 230 people in the sample having a degree is: \begin{lstlisting}
n=800
p=0.286
m=n*p
sd=sqrt(n*p*(1-p))
pn=
pnorm(230+0.5,m,sd)
-pnorm(220-0.5,m,sd)
\end{lstlisting}
Which evaluates to $0.3194833$. The exact probability is given by \begin{lstlisting}
pb=pbinom(230,n,p)-pbinom(220,n,p)
\end{lstlisting} which evaluates to $0.2959644$. The error in the approximation is \code{(pn-pb)/pb} which evaluates to $\approx 8\%$.


\setsection{15}
By definition, $W \sim \chi_n^2$. The sampling distributions of $W_n$ for $n\in\{2,3,4,5\}$ are shown in the accompanying figure.
\img{15.pdf}
The means and variances are tabulated as:
\begin{table}[h]
\centering
\begin{tabular}{@{}rrr@{}}
\toprule
\multicolumn{1}{l}{$n$} & \multicolumn{1}{l}{$\mu$} & \multicolumn{1}{l}{$\sigma^2$} \\ \midrule
2                     & 2.02                  & 4.07                  \\
3                     & 2.99                  & 5.82                  \\
4                     & 3.98                  & 7.80                  \\
5                     & 4.99                  & 10.01                 \\ \bottomrule
\end{tabular}
\end{table}


From the table, we can see that $\mu=n$ and $\sigma^2=2n$.

\setsection{18}
\subsection{}
The sampling distribution is shown below:\img{18.pdf}
\subsection{}
The mean from sample is $3.00$ with standard error $0.55$.
From CLT, we expect $\mu_{\overline{X}}=\mu=(1/3)^{-1}=3$ and $\sigma_{\overline{X}}=\sigma/\sqrt{n}=(1/3)^{-1}/\sqrt{30}\approx 0.55$, giving a difference of $<1\%$.
\subsection{}
We get $P=0.83$.
\subsection{}
We get $P=0.82$, which is only $\approx 1\%$ off.


\setsection{20}
\subsection{Distribution of Sample Minimum}
\begin{align*}
&&P(X_{\text{min}} \ge x) &= \prod_{i=1}^n P(X_i \ge x) \\
&\implies& 1-\Fmin(x) &= (1-F(x))^n\\
&\implies& -\fmin(x) &= n(1-F(x))^{n-1} (-f(x))\\
&\implies& \fmin(x) &= n\left(1-F(x)\right)^{n-1}f(x) \qed
\end{align*}
\subsection{Distribution of Sample Maximum}
\begin{align*}
&&P(X_{\text{max}} \le x) &= \prod_{i=1}^n P(X_i \le x) \\
&\implies& \Fmax(x) &= F^n(x)\\
&\implies& \fmax(x) &= nF^{n-1}(x)f(x) \qed
\end{align*}

\setsection{22}
\subsection{}
$F(x)=\intg{1}{x}{\frac{2}{x^2}}{x}=2-2/x$, so from previous result $\fmax=2\left(2-\frac{2}{x}\right)^{2-1}\frac{2}{x^2}=2\left(2-\frac{2}{x}\right)\frac{2}{x^2}$.
\subsection{}
$E_{X_{\text{max}}}=\intg{1}{2}{x\fmax(x)}{x}\approx 1.55.$

\setsection{25}
\begin{align*}
f_{X_1+X_2}(x)&=P(X_1+X_2=x)\\
&=\sum_{i=0}^x\left(
	\frac{\lo^i}{i!}e^{-\lo}
	\frac{\lt^{x-i}}{(x-i)!}e^{-\lt}
\right)\\
&=e^{-(\lo+\lt)}\sum_{i=0}^x\frac{
	\lo^i\lt^{x-i}
}{
	i!(x-i)!
}\\
&=\frac{e^{-(\lo+\lt)}}{x!}
\sum_{i=0}^x\left(
	\frac{x!}{i!(x-i)!}
	\lo^i\lt^{x-i}
\right)\\
&=\frac{e^{-(\lo+\lt)}}{x!}
\sum_{i=0}^x\left(
	\binom{x}{i}
	\lo^i\lt^{x-i}
\right)\\
&=\frac{e^{-(\lo+\lt)}}{x!}(\lo+\lt)^x\\
&=f_Z(x)\text{, where }Z\sim\mathcal{P}(\lo+\lt)
\end{align*}\begin{align*}
&\implies X\sim\mathcal{P}\left(\sum_{i=1}^{10}\lambda_i\right)\\
&\implies X\sim\mathcal{P}(30)\\
&\implies f_X(x)=\frac{30^x}{x!}e^{30}
\end{align*}

\setsection{27}
\subsection{} The distribution seems quite close to normal.
\subsection{}
Theoretical mean $\mu_{\overline{X}}=\mu=10.17$. Actual mean is $10.02$. Difference is $\approx1.5\%$.
\subsection{}
Theoretical s.e. $\sigma_{\overline{X}}=\frac{\sigma}{n}\sqrt{\frac{N-n}{N-1}}\approx4.55$. Actual s.e. $4.56$. Difference is $<1\%$.
\subsection{}
The differences reduce with increasing $n$.

\setsection{28}
It can be seen from the plots below that the distribution of sample variances is approximately normal, and the fit to normal improves with sample size.
\img{28a.pdf}
\img{28b.pdf}
\img{28c.pdf}
\img{28d.pdf}
\end{document}